\section{Transport Protocol}


\subsection{Channel Multiplexing}

SSTP supports multiple channels within a single encrypted session. It provides
both reliable and unreliable data delivery.
We borrow and simplify aspects of QUIC streams to implement a reliable stream
abstraction. 

An SSTP message is composed of a SessionID, a Counter, Encrypted Data, and MAC.
We encapsulate channel \emph{frames} within the Encrypted Data section of the
SSTP message, providing authentication and encryption by default for all
packets. For simplicitly, we only one frame is contained in one SSTP message,
although future extensions may allow multiple frames.

All SSTP channel frames are one of four types: 
\begin{itemize}
    \item Channel Initiate Request
    \item Channel Initiate Response
    \item Data
    \item Control
\end{itemize}
Once an encrypted SSTP session has been established, either the client or the
server can request to initiate a shared channel. A channel
handshake is performed to establish the channel. In the absence of the loss
of a handshake message, channel initiation handshakes will complete in one round
trip. We do not need a TCP-like 3-way handshake
because initial sequence numbers (frame counters) for both direcitons of
communication start at 0. Each channel has a channel ID, so there is no
confusion between separate channels, and spoofing attacks from non-random
sequence numbers are already prevented from the encrypted transport protocol. We show the protocol in Figure~\ref{fig:channel_protocol}.

% Diagram here
\begin{figure}[h]
	\centering
	\includegraphics[width=0.6\columnwidth]{figures/channel_protocol.pdf}
	\caption{\textbf{Channel Protocol}---%
	\textnormal{}
	}
	\label{fig:channel_protocol}
	\vspace{-5pt}
\end{figure}

\vspace{5pt} \noindent \textit{Channel Initiate Request:}\quad
The first message from the initiator to the
responder contains a Channel Initiate Request Frame. We show the structure of
the Channel Initiate Request Frame in Figure~\ref{fig:initiate_request_frame}. 

The frame header consists of the Channel ID and the Type. The Mode indicates
whether the channel will be unreliable or reliable. Handshake frames also
consist of a maximum frame size and an initial window size that the initiator is
willing to handle. Channels also denote the
\emph{Action} that will be performed (e.g., file transfer,
remote shell). Action labels will be a value agreed upon by both
the client and the server (or standardized).

\input{bytefields/initiate_request_frame.tex}

\vspace{5pt} \noindent \textit{Channel Initiate Response:}\quad 
The second message (responder to initiator) contains a Channel Initiate Response Frame. We show the structure of the
Initiate Response Frame below.

\vspace{2ex}
\begin{small}
\begin{bytefield}[bitwidth=7pt,bitheight=15pt]{32}
    \bitheader{0-31} \\
    \bitbox{32}{ channel ID } \\
    \bitbox{8}{type := 0x1} &
    \bitbox{8}{response} &
    \bitbox{16}{max frame size} \\
    \bitbox{32}{initial window size } \\
    \bitbox{16}{action label}
    \bitbox{16}{initial action data size} \\
    \bitbox{32}{initial action data} \\
\end{bytefield}
\end{small}

\vspace{5pt} \noindent \textit{Data:}\quad 
Data frame structure below.

\vspace{2ex}
\begin{small}
\begin{bytefield}[bitwidth=7pt,bitheight=15pt]{32}
    \bitheader{0-31} \\
    \bitbox{32}{ channel ID } \\
    \bitbox{8}{type := 0x3} &
    \bitbox{24}{reserved := 0x0} \\
    \wordbox{2}{frame CTR} \\
    \bitbox{16}{maximum ack delay} &
    \bitbox{16}{action data size} \\
    \bitbox{32}{action data} \\
\end{bytefield}
\end{small}


\vspace{5pt} \noindent \textit{Control:}\quad
Control frame structure below.

\vspace{2ex}
\begin{small}
\begin{bytefield}[bitwidth=7pt,bitheight=15pt]{32}
    \bitheader{0-31} \\
    \bitbox{32}{ channel ID } \\
    \bitbox{8}{type := 0x3} &
    \bitbox{16}{control type} &
    \bitbox{8}{reserved := 0x0} \\
    \bitbox{16}{control data size} &
    \bitbox{16}{ack range count} \\
    \bitbox{32}{control data} \\
\end{bytefield}
\end{small}

\subsubsection{Reliable Channels}











