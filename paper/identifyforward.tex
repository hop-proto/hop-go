\section{Granting Authorization}
In this section, we present a protocol for authorization grants \todo{secure
delegation?} with SSTP. One of the goals of SSTP is to provide first-class support for the ability
for a principal (the user) to grant a third-party client permission to perform some action on a
target server. 
Existing protocols such as SSH do not support such use cases, and workarounds such as ssh-agent
forwarding are insecure~\cite{kogan2017guardian-agent}.
We abide by the Secure Delegation Principle defined by Kogan et al.~\cite{kogan2017guardian-agent}, which allows a client to act under a principal's authority only after the principal can verify
and enforce the client's intent. This intent includes the identity of the client used for the
authorization grant, a single command the client can issue, and to whom the client
can issue the command. 


\subsection{Authorization Grant Protocol}

Broadly speaking, the process works as follows. We assume that the client has already been
authenticated to the principal and shares an SSTP connection with it. The client requests that the principal grants permission for it to perform an
action on the server, i.e., the \emph{intent}. If the principal approves the intent
request, it establishes an SSTP tunnel with the server, proxied through the client. The principal
communicates the identity and intent of the client to the server over the SSTP tunnel. If the
server also approves the intent, the client can then establish its own SSTP tunnel with the server
using the identity approved by the principal and execute the specified action. The server enforces
that any action executed by the client (using the current identity) is performed at most once.

\begin{figure}
    \includegraphics[width=1\columnwidth]{figures/authorization_grant.pdf}
    \vspace{-5pt}
    \caption{\textbf{Authorization Grant Protocol}---%
    \textnormal{}
    }
    \label{fig:authorization_grant}
    \end{figure}

\subsubsection{Intent Request}
The client first makes an intent request for the principal to approve or deny. To do so, 
it initiates a reliable Authorization Grant Channel with the Principal, and sends an \emph{Intent Request} over this
channel. The message data contains a type identifier indicating that it is an Intent Request, and
includes an \emph{Intent Message}. The Intent Message is a serialization of the following fields: the name and
port of the destination server, a hash of the ephemeral public key used as the client's temporary
identifier, a single channel type, and any associated data for the channel type (e.g., a
command to be run). The principal checks this Intent Request against its security policy and either approves
or denies the request. If it denies the request, it will reply with an Intent Denied message
indicating the reason it was denied.

\subsubsection{Intent Communication}
If the principal approves the Intent Request, it communicates the intent to the server in order to
grant the client permission to establish an SSTP tunnel with the server. The server will then also
be able to enforce that the client only executes the actions specified in the Intent Request.
However, the principal may not necessarily have connectivity to the server. Therefore,
the principal will initiate a Network Proxy channel with the client, asking it to proxy a connection to the server. The client forwards
any messages sent through the Network Proxy channel to the server over UDP, and any replies
from the server are forwarded back to the principal. The principal then establishes an SSTP
tunnel with the server and initiates an Authorization Grant Channel. It relays the Intent Message to the
server via an \emph{Intent Communication} message over this channel, which simply contains a type
identifier and the Intent Message as the payload. At this
point, the server is aware of the client's identity, the actions it wishes to execute, and who is
granting them authorization. It then replies with either an Accept
Intent Communication message or a Deny Intent Communication message. If it accepts, the server
sets a timer equal to the length of an SSTP handshake timeout \todo{make sure
this is defined somewhere} for the client to successfully complete an SSTP handshake with the
server. Note that
the communication between the principal and the server is end-to-end by virtue of SSTP, so any
Intent Communication messages cannot be read or tampered by the client. 

\subsubsection{Perform Intent}
Once the server has accepted the intent, the principal sends a \emph{Perform Intent} message to the
delegate via the Authorization Grant Channel from before. This message is simply a signal to the client
that it now has permissions to execute the requested actions on the server and only contains a message type identifier. The client will then establish its own SSTP
tunnel with the server, using the ephemeral public key specified in the Intent Request as its client
static in the SSTP handshake protocol. It can then execute its actions as specified in the Intent
Request. The principal can now tear down any Authorization Grant Channels created as part of the
authorization grant, and end the SSTP session with the server. The
protocol connection flow is shown in Figure~\ref{fig:authorization_grant}.