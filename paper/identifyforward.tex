\section{Identify Forwarding}

In this section, we present a protocol for secure identity forwarding \todo{secure delegation?}. The purpose of delegation is
for the user (the principal) to grant a party (the delegate) permission to perform some operation on
a target (the server). We
abide by the Secure Delegation Principle defined by Kogan et al.~\cite{kogan2017guardian-agent},
which allows a delegate to act under a principal's authority only after the principal can verify
and enforce the delegate's intent. This intent includes the identity of the delegate used for the
delegation instance, a set of commands or operations the delegate can issue, to whom the delegate
can issue these commands, and when the delegate can issue them. 




\subsection{Delegation Protocol}

Broadly speaking, the process works as follows. We assume that the delegate has already been
authenticated to the principal and shares an SSTP connection with it. The delegate requests that the principal grants permission for it to perform a
series of actions on the server, i.e., the \emph{intent}. If the principal approves the intent
request, it establishes an SSTP tunnel with the server, proxied through the delegate. The principal
communicates the identity and intent of the delegate to the server over the SSTP tunnel. If the
server also approves the intent, the delegate can then establish its own SSTP tunnel with the server
using the identity approved by the principal and execute the specified actions.

\begin{figure}
    \includegraphics[width=1\columnwidth]{figures/delegation_protocol.pdf}
    \vspace{-5pt}
    \caption{\textbf{Secure Delegation Protocol}---%
    \textnormal{}
    }
    \label{fig:delegation_protocol}
    \end{figure}

\subsubsection{Intent Request}
The delegate first makes an intent request for the principal to approve or deny. To do so, 
it initiates a reliable Delegation Channel with the Principal, and sends an \emph{Intent Request} over this
channel. The message data contains a type identifier indicating that it is an Intent Request, and
includes an \emph{Intent Message}, which is a serialization of the following fields: the ID, address, and port of the destination
server, an ephemeral public key used as the delegate's temporary identifier, a list of allowed
channel types and operations specific to those channels (e.g., commands to be run over a shell
channel), and a timestamp for when the delegate's permissions expires. \todo{how to dictate when the
delegate can issue commands. Should it be time based, number of uses, something else?} The principal checks this Intent Request against its security policy and either approves
or denies the request. If it denies the request, it will reply with an Intent Denied message
indicating the reason it was denied. \todo{do we really need an Accept Intent Request message type?
Maybe for symmetry, but it doesn't seem to be required}.

\subsubsection{Intent Communication}
If the principal approves the Intent Request, it communicates the intent to the server in order to
grant the delegate permission to establish an SSTP tunnel with the server. The server will then also
be able to enforce that the delegate only executes the actions specified in the Intent Request.
However, the principal may not necessarily have connectivity to the server. Therefore,
the principal will initiate a Network Proxy channel with the delegate, asking to it to proxy a connection to the server. The delegate forwards
any messages sent through the Network Proxy channel to the server over UDP, and any replies
from the server are forwarded back to the principal. The principal then establishes an SSTP
tunnel with the server and initiates a Delegation Channel. It relays the Intent Message to the
server via an \emph{Intent Communication} message over this channel, which simply contains a type
identifier and the Intent Message as the payload. The server then replies with either an Accept
Intent Communication message or a Deny Intent Communication message, specifying a reason. At this
point, the server is aware of the delegate's identity, the actions it wishes to execute, and when it
will execute them. The server adds the delegate's ephemeral private key to its list of authorized
keys along with its expiration time. Note that
the communication between the principal and the server is end-to-end by virtue of SSTP, so any
Intent Communication messages cannot be read or tampered by the delegate. 

\subsubsection{Perform Intent}
Once the server has accepted the intent, the principal sends a \emph{Perform Intent} message to the
delegate via the Delegation Channel from before. This message is simply a signal to the delegate
that it now has permissions to execute actions on the server and only contains a message type identifier. The delegate will then establish its own SSTP
tunnel with the server, using the ephemeral public key specified in the Intent Request as its client
static in the SSTP handshake protocol. It can then execute its actions as specified in the Intent
Request. The server can optionally prove to the principal that it received valid delegated
permissions for the actions via the proxied Delegation Channel with the principal. Otherwise, the
principal and tear down any Delegation Channels created as part of the delegation session. The
protocol connection flow is shown in Figure~\ref{fig:delegation_protocol}.